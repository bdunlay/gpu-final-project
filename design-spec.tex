\title{K-Means Parallelization \\
	Final Project Design Specification}
\author{Brian Dunlay\\
	EE590A - Winter 2017
}
\date{\today}

\documentclass[11pt]{article}
\usepackage{amsfonts}
\usepackage{mathtools}
\usepackage{listings}
\usepackage{color}

\lstset{frame=tb,
  language=C++,
  aboveskip=3mm,
  belowskip=3mm,
  showstringspaces=false,
  columns=flexible,
  basicstyle={\small\ttfamily},
  numbers=none,
  numberstyle=\tiny\color{gray},
  keywordstyle=\color{blue},
  commentstyle=\color{green},
  stringstyle=\color{red},
  breaklines=true,
  breakatwhitespace=true,
  tabsize=3
}




\begin{document}
\maketitle

\section{K-Means Equations}

K-means is a clustering algorithm. Given an integer $k$ and a set of $n$ data points $X \subset \mathbb{R}^d$ we aim to choose $k$ centers $C$ so as to minimize the following function \cite{arthur}:
\newline

\begin{math}
\phi = \displaystyle\sum_{x \in X} min_{c \in C }\| x - c \|^2
\end{math}

\subsection{Iterative Algorithm}
The algorithm can be implemented in three steps:

\begin{enumerate}
\item Arbitrarily choose an initial k centers $C = {c_1, c_2, \cdots, c_k}$
\item For each $i \in {1, \cdots, k}$, set the cluster $C_i$ to be the set of points in $X$ that are closer to $c_i$ than they are to $c_j$ for all $j \neq i$
\item For each $i \in {1, \cdots, k}$, set $c_i$ to be the center of mass of all points in $C_i$: $c_i = \frac{1}{|C_i|}\sum_{x \in C_i}x$
\item Repeat Steps 2 and 3 until $C$ no longer changes
\end{enumerate}

\section{Sequential Reference}

I wrote a sequential reference in C++ in order to achieve a baseline. I found that
it is less computationally intensive to convert the image to the YUV color space and
do a comparison (euclidean distance) between points than it would have been to do
the same in RGB colorspace. 

\subsection{Sample Image}

\begin{figure}
    \centering
    \includegraphics[width=0.6\textwidth]{fruit.png}
    \caption{Image\cite{fruit} before segmentation}
    \label{fig:fruit}
\end{figure}

\begin{figure}
    \centering
    \includegraphics[width=0.6\textwidth]{fruit-segmented.png}
    \caption{Image after segmentation}
    \label{fig:fruit-segmented}
\end{figure}

This image was segmented based on 5 coordinates hand-picked from within the 
image. The coloration is a fixed luma value (Y) and the average chroma value (U and V)
for the clusters that were found. The average runtime over 30 runs for processing this 
image was 3.026 seconds. 

\subsection{Code}

See \emph{sequential-kmeans.cpp} for the sequential code reference.

\section{Complexity Analysis}

\section{Parallel Pseudo-Code}

\begin{thebibliography}{9}

\bibitem{arthur}
  David Arthur and Sergei Vassilvitskii,
  \emph{k-means++: The Advantages of Careful Seeding},
  http://ilpubs.stanford.edu:8090/778/1/2006-13.pdf

\bibitem{fruit}
  Didriks,
  Fruit Salad,
  https://flic.kr/p/a6W1Te

\end{thebibliography}

\end{document}
This is never printed
